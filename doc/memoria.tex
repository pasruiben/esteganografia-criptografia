\documentclass[12pt]{article}

\usepackage[spanish,activeacute]{babel}
\usepackage[utf8]{inputenc}
\usepackage{graphicx}
%\usepackage{fullpage}

\title{Ocultar datos en archivos de sonido}
\author{Juan Antonio Cano Salado \and Borja Moreno Fernández \and Pascual Javier Ruiz Benítez}

\begin{document}

\maketitle

\newpage
\tableofcontents

\newpage
\section{Resumen}

Este trabajo aborda la ocultación de datos en archivos de sonido o, lo que es lo mismo, la esteganografía de audio. El objetivo de esta disciplina consiste básicamente en ocultar información (de cualquier tipo) en archivos de sonido, de forma que los cambios llevados a cabo en el archivo original resulten imperceptibles para una persona.

Comenzaremos estudiando los principios básicos de la esteganografía en general, y de la esteganografía de audio en particular. A continuación, analizaremos algunos de los métodos más comúnmente utilizados en esteganografía de audio. Hecho esto, estudiaremos uno de los formatos de audio más populares: el formato WAV. Finalmente, incluiremos una descripción de la implementación realizada y proporcionaremos un manual de instalación y manejo de la aplicación desarrollada.

\section{Introducción}

\subsection{Esteganografía}

La esteganografía es la disciplina en la que se estudian y aplican técnicas que permiten el ocultamiento de mensajes u objetos, dentro de otros, llamados portadores, de modo que no se perciba su existencia.

Los orígenes de la esteganografía datan de la antigua Grecia. Heródoto, famoso historiador griego, informa del uso de la esteganografía en informes de Grecia a Persia. El método consistía en afeitar la cabeza de un esclavo y tatuar allí un mensaje. Dicho mensaje quedaba oculto cuando el pelo del esclavo volvía a crecer. Para leer el mensaje sólo era necesario volver a afeitarle la cabeza al esclavo. La idea era que nadie sospechase de la existencia de dicho mensaje.

La esteganografía ha sido usada en numerosas ocasiones a lo largo de la historia: tintas invisibles, acrósticos o mensajes microscópicos son sólo algunas de sus formas. Recientemente, en plena era digital, la estaganografía ha adquirido gran importancia como tecnología utilizada en el campo de la seguridad informática. Pueden ocultarse mensajes secretos en correos electrónicos, imágenes, audio e incluso vídeo.

Diversos grupos han mostrado tener un gran interés en las aplicaciones de la esteganografía. Algunos, interesados en la protección de derechos de autor. Otros, preocupados por proteger la privacidad de sus mensajes. Muchos gobiernos temen que la esteganografía podría convertirse en una herramienta de gran utilidad para criminales y grupos terroristas. En cualquier caso, parece claro que existen multitud de aplicaciones y usos para la esteganografía, y la mayoría de los expertos están de acuerdo en que será un tema de gran interés durante los próximos años.

La figura \ref{prisioneros} ilustra un ejemplo simple de un problema en el que el uso de la esteganografía puede ser de utilidad.

\begin{figure}[h]
  \centering
    \includegraphics[width=\textwidth]{img/illustration2}
  \caption{Problema de los prisioneros}
  \label{prisioneros}
\end{figure}

Dos prisioneros en una cárcel necesitan comunicarse para ultimar los detalles de un plan de fuga, pero tienen un gran problema: el guardia de la prisión tiene acceso a toda la correspondencia entre los prisioneros. La criptografía por sí sola no es una solución. Un mensaje cifrado levantaría todo tipo de sospechas. Los prisioneros deben idear un sistema que les permita pasar mensajes de apariencia inocente, con información oculta que sólo ellos puedan entender. Esto es exactamente lo que persigue la esteganografía.

La esteganografía puede además combinarse con la criptografía para crear sistemas más seguros. Así, distinguimos:

\begin{itemize}

\item Esteganografía pura

La fortaleza del sistema recae en los algoritmos de ocultación y extracción de la información, que solo el emisor y el receptor del mensaje deberían conocer.

\item Esteganografía de clave privada

Fruto de la combinación de esteganografía pura con criptosistemas simétricos. Se asume que un atacante podría conocer los algoritmos de ocultación y extracción de la información. Por este motivo, el mensaje se cifra utilizando un cifrado simétrico antes de ocultarlo. De esta manera, incluso si el atacante intercepta la transmisión y logra extraer la información aún tendrá que enfrentarse al criptoanálisis del criptosistema utilizado.

\item Esteganografía de clave pública

Basada en unir esteganografía pura y criptosistemas de clave pública. De esta manera, el emisor y el receptor evitan tener que compartir una clave privada.

\end{itemize}

\subsection{Audio digital}

El audio digital se diferencia del sonido analógico tradicional en que es una señal discreta en lugar de una señal continua. Esta señal discreta es creada llevando a cabo un proceso de muestreo y cuantización de una señal analógica continua. La frecuencia de muestro varía según los propósitos. Por ejemplo, la frecuencia de muestreo estándar para un CD de audio digital es de alrededor de 44kHz. La figura \ref{audiodigital} ilustra el proceso de muestro y cuantización para producir una señal de audio digital a partir de una señal continua de audio analógico.
En dicha figura se ha exagerado la naturaleza discreta de la señal digital. Sin embargo, las frecuencias de muestreo usuales permiten que el audio digital sea prácticamente idéntico a la señal de audio analógica original.

\begin{figure}[h]
  \centering
    \includegraphics[width=\textwidth]{img/digitalaudio}
  \caption{Audio digital}
  \label{audiodigital}
\end{figure}

Los archivos de audio digital se almacenan en el ordenador como una secuencia de ceros y unos. Con las herramientas apropiadas, es posible alterar individualmente los bits que componen estos archivos. Esto hará posible llevar a cabo modificaciones a la secuencia binaria que produzcan cambios en la señal de audio que no sean perceptibles al oído humano.

\subsection{Esteganografía de audio}

En un sistema de estaganografía digital informatizado, se ocultan mensajes secretos en audio digital. El mensaje secreto se introduce alterando ligeramente la secuencia binaria del archivo de audio. Existen aplicaciones de esteganografía de audio que permiten ocultar mensajes en archivos de audio en formato WAV, AU e incluso MP3.

Ocultar mensajes secretos en archivos de sonido suele ser más difícil que ocultar mensajes en archivos de imagen o vídeo. Esto se debe a la mayor sensibilidad que presenta en las personas el sistema auditivo frente al sistema visual. Existen un buen número de métodos y algoritmos para llevar a la práctica estos conceptos, algunos de ellos sencillos y otros más complejos (basados en técnicas avanzadas de procesado digital de señales). Estos métodos serán analizados en mayor detalle en la sección ``Soluciones presentadas".

\section{Descripción del problema}

\section{Soluciones presentadas}

En esta sección presentamos algunos de los métodos más comúnmente utilizados en esteganografía de audio. Pueden encontrarse implementaciones de estos métodos en la Web. Algunos de los métodos más avanzados requieren conocimiento previo de técnicas de procesado de señales, análisis de Fourier y otras áreas matemáticas. Se han preferido utilizar diagramas y pseudocódigo en lugar de fórmulas matemáticas exactas para intentar hacer la teoría más accesible a lectores con conocimientos básicos de esteganografía.

\subsection{Codificación en el bit menos significativo (LSB Coding)}

Es el método más simple de inclusión de información en un archivo de audio digital. Se basa en sustituir el bit menos significativo de cada sample (porción de sonido) con un bit del mensaje a ocultar. Este método permite ocultar grandes cantidades de información. La figura \ref{lsbcoding} ilustra cómo se oculta el mensaje ``HEY'' en un archivo de audio compuesto por samples de 16 bits.

\begin{figure}
  \centering
    \includegraphics[width=\textwidth]{img/lsbimage}
  \caption{LSB Coding}
  \label{lsbcoding}
\end{figure}

Algunas implementaciones de este método utilizan los dos (o más) bits menos significativos de cada sample para ocultar información. Esto hace que la cantidad de información que se puede insertar aumente, pero también eleva la cantidad de ruido que es introducida en la señal de audio resultante. Por ello, es importante considerar el tipo de señal de audio antes de decidir el número de bits por sample a utilizar. Por ejemplo, un archivo de sonido grabado en una estación de metro con gran cantidad de ruido de fondo podría soportar la inclusión de dos o más bits de información en cada sample, ya que el ruido de fondo enmascararía el ruido introducido en el proceso de ocultación de información. Por contra, un archivo de sonido que contenga un solo de piano será mucho más sensible a estas modificaciones, por lo que el uso de más de un bit por sample podría hacer que las modificaciones resultaran perceptibles al oído humano.

Para poder extraer un mensaje secreto codificado utilizando LSB, el receptor necesita tener acceso a la secuencia de samples utilizada en el proceso de inserción de información. Normalmente, la longitud del mensaje secreto a ocultar es mucho menor que el número total de samples del archivo de sonido. Una decisión a tomar consiste por tanto en cómo elegir qué subconjunto de samples del archivo original se usarán para contener información. El emisor y el receptor deben ponerse de acuerdo en esta cuestión. Una técnica trivial consiste en comenzar en el inicio del archivo de sonido y llevar a cabo la inserción de información en los primeros samples, hasta que el mensaje ha sido completamente introducido. A partir de ahí, el resto de samples quedarían inalterados. El problema de esta técnica radica en que la primera parte del archivo de sonido tendrá propiedades estadísticas diferentes a la del resto del archivo, que no fue modificado. Una solución a este problema es añadir al mensaje secreto una secuencia de bits aleatorio de manera que la longitud total del mensaje fuera igual al número total de samples. El inconveniente es que el proceso de inserción modificaría ahora muchos más samples de los que la transmisión del mensaje original requería, lo cual hace que sea más sencillo detectar la existencia del mensaje oculto.

Una solución más sofisticada se basa en usar un generador de números pseudoaleatorios para distribuir el mensaje entre los samples del archivo de sonido. Para poder hacer esto, el emisor y el receptor deben ponerse de acuerdo en la semilla que utilizarán para generar la secuencia pseudoaleatoria de samples a utilizar. De esta manera el receptor puede reconstruir la secuencia y averiguar cuáles son los samples que contienen el mensaje oculto. Hay que tener cuidado con que el generador produzca el mismo sample dos veces. Si esto ocurriese, se modificaría el bit menos significativo de un sample cuyos bits menos significativos ya habían sido modificados, produciéndose una colisión. Una manera de evitar esto es almacenar en una lista los samples que ya han sido modificados, de manera que si el generador indica que se vuelva a utilizar un sample ya empleado, se ignora. Otra alternativa consiste en calcular una permutación pseudoaleatoria del conjunto de samples del archivo de audio, y seleccionar los primeros. Con esta técnica es imposible que se utilice el mismo sample más de una vez.

\subsection{Codificación de paridad (Parity Coding)}

En lugar de dividir la señal en samples individuales, este método divide la señal en regiones disjuntas. Una región es un conjunto de samples, de un tamaño determinado. Todas las regiones tendrán el mismo tamaño, excepto quizás la última, si el número de samples del archivo no es múltiplo del número de samples por región. El método oculta un bit del mensaje secreto en el bit de paridad de cada región. El bit de paridad de una región se define como el resultado de aplicar la operación lógica XOR a todos los bits de todos los samples de la región. Si el bit de paridad de la región seleccionada (aquella en la que se ha decidido ocultar un bit del mensaje) no coincide con el bit secreto a incrustar, se modifica el bit menos significativo de uno de los samples de la región (no importa cual). Si coincide, no es necesario llevar a cabo ninguna modificación. De esta manera, el emisor tiene una serie de samples entre los que elegir a la hora de ocultar cada bit.

\begin{figure}
  \centering
    \includegraphics[width=\textwidth]{img/parity}
  \caption{Parity Coding}
  \label{paritycoding}
\end{figure}

La figura \ref{paritycoding} ilustra cómo se ocultan, usando este método, los tres primeros bits del mensaje ``HEY'' (010, ya que en formato ASCII la `H'\space se representa por el número 72 en decimal, 01001000 en binario).

Para poder llevar a cabo el proceso de extracción de la información el receptor debe conocer el tamaño de las regiones así como el orden en el que éstas fueron utilizadas a la hora de ocultar la información. Como en el método anterior, puede optarse por utilizar las primeras regiones del archivo de sonido de manera secuencial, o bien por usar un generador pseudoaleatorio que, a partir de una semilla, determine el orden en que deben escogerse las regiones en las que ocultar información. En caso de utilizar esta segunda alternativa, la semilla y el generador pseudoaleatorio deben ser compartidos por emisor y receptor, evidentemente.

El proceso de obtención de la información oculta se limita a calcular los bits de paridad de las regiones empleadas durante el proceso de inserción del mensaje. La concatenación de estos bits de paridad dará como resultado el mensaje original. Como puede observarse, el receptor no necesita conocer en ningún momento qué samples concretos se han visto modificados con respecto al archivo de sonido original.

\bigskip

Los métodos de codificación en el bit menos significativo y codificación de paridad presentan dos desventajas fundamentales. En primer lugar, ambos introducen ruido en la señal original, y el oído humano es muy sensible al ruido, a menudo siendo capaz de detectar incluso las cantidades más insignificantes. El método de codificación de paridad es ligeramente mejor en este aspecto, ya que asegura que sólo se modificará como máximo un bit en cada región. La otra gran desventaja que presentan estos métodos es que no son robustos. Pequeñas modificaciones en el archivo de sonido que contiene el mensaje secreto podrían destruir dicho mensaje. La robustez de estos métodos puede aumentar en cierta medida si se utilizan técnicas de redundancia, ocultando cada bit del mensaje secreto en dos o más localizaciones del archivo de audio. El problema que presenta esta solución es que aumenta la cantidad de bits a incrustar en el archivo de sonido original, lo cual reduce el tamaño máximo del mensaje que se puede ocultar en una señal de audio determinada a la vez que se incrementa la probabilidad de que las modificaciones realizadas resulten perceptibles.

\section{Conclusiones}

\section{Problemas abiertos}

\section{Implementación realizada}

\section{Manual de instalación y manejo de la aplicación}

\section{Bibliografía}

\section{Tabla de tiempos}

\end{document}